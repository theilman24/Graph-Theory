\documentclass[12pt]{article}
\usepackage{amsmath}
\usepackage{amssymb}


\pagestyle{headings} \thispagestyle{empty}
%\pagestyle{empty}
  \textwidth      6.4in
      \oddsidemargin 0.0in
      \topmargin     -0.4in
      \topskip          0pt
      \headheight      12pt
      \footskip        18pt
%      \footheight      12pt
      \textheight     650pt

\parindent=0cm
\baselineskip=2cm

%\include these lines if you want to use the LaTeX "theorem" environments
\newtheorem{theorem}{Theorem}[section]
\newtheorem{definition}[theorem]{Definition}
\newtheorem{lemma}[theorem]{Lemma}
\newtheorem{corollary}[theorem]{Corollary}
\newtheorem{guess}{Conjecture}
\newtheorem{example}[theorem]{Example}

%include lines like this if you want to define your own commands
%to save typing
\newcommand{\PROOF}{\noindent {\bf Proof}: }
\newcommand{\REF}[1]{[\ref{#1}]}
\newcommand{\Ref}[1]{(\ref{#1})}
\newcommand{\dt}{\mbox{\rm   dt}}
\newcommand{\qed}{\Large{\bf{$\diamond$}}}
\newcommand{\phat}{\hat{p}}

\DeclareSymbolFont{AMSb}{U}{msb}{m}{n}
\DeclareMathSymbol{\N}{\mathbin}{AMSb}{"4E}
\DeclareMathSymbol{\Z}{\mathbin}{AMSb}{"5A}
\DeclareMathSymbol{\R}{\mathbin}{AMSb}{"52}
\DeclareMathSymbol{\Q}{\mathbin}{AMSb}{"51}
\DeclareMathSymbol{\I}{\mathbin}{AMSb}{"49}
\DeclareMathSymbol{\C}{\mathbin}{AMSb}{"43}

%\setstretch{1.5}

\renewcommand{\baselinestretch}{1.5}

\begin{document}

\textbf{Name: Taylor Heilman}    \hspace{4in} 
\begin{center} \textbf{CS 275: Spring 2016} \end{center}

\begin{enumerate}

\item Combinatorial Argument Proof for Theorem 1.1 ( The sum of the entries of row $n$ of Pascal's triangle is $2^n$)

{PF

Notice $(a+b)^n = (a+b)(a+b).......(a+b)$, where $(a+b)$ is written out $n$ times
Example: $(a+b)^4 = (a+b)(a+b)(a+b)(a+b)$

The terms in $(a+b)^n$ have the form of $a^{n-k}b^k$ where $k$ is a number between $0$ and $n$

The coefficient of $a^{n-k}b^k$ for some $k$ is the number of ways to choose $k$ factors of $b$ from $n$ factors of $(a+b)$ 

The factors of $a$ come from the remaining $(n-k)$

Example: $a^4 = 1 (aaaa)$
	       
$a^3b^1 = 4 (aaab)(aaba)(abaa)(baaa)$

$a^2b^2 = 6 (aabb)(abba)(bbaa)(baba)(baab)(abab)$

This can be written as $ {n \choose k}$

Therefore $(a+b)^n = \sum_{k=0}^{n} {n \choose k} a^{n-k}b^k$

//}

\item Induction Proof

{
PF

Let $n=1$

$(a+b)^1 =  {1 \choose 0}a^{1-0}b^0 + {1 \choose 1}a^{1-1}b^1 = \sum_{k=0}^{1} {1 \choose k} a^{n-k}b^k$

Assume $(a+b)^{t} = \sum_{k=0}^{t} {t \choose k} a^{t-k}b^k$

We show that $(a+b)^{t+1} = \sum_{k=0}^{t+1} {t+1 \choose k} a^{(t+1)-k}b^k$

 $(a+b)^{t+1} = (a+b)^t (a+b)$
 
 $= \sum_{k=0}^{t} {t \choose k} a^{t-k}b^k (a+b)$
 
 $= \sum_{k=0}^{t} {t \choose k} a^{t-k}b^k (a+b)$

$=a\sum_{k=0}^{t} {t \choose k} a^{t-k}b^k  + b\sum_{k=0}^{t} {t \choose k}a^{t-k}b^k  $

$=\sum_{k=0}^{t} {t \choose k} a^{t-k+1}b^k  + b\sum_{k=0}^{t} {t \choose k}a^{t-k}b^k+1  $

according to the summations, the coefficient of $a^{t+1-k}b^k$  is  ${t \choose k} + {t \choose k-1}$

This can be rewritten as ${t + 1 \choose k}$ by Formula 1.6

By the Principle of Mathematical Induction, the Formula is proven.



//
}

\item Induction Proof for Theorem 1.1

{PF

Formula 1.6$ {n \choose k} = {n-1 \choose k-1} + {n-1 \choose k}$


Let $n=0$

$\sum_{k=0}^{0} {0 \choose k} = 2^0 = 1$

Assume $n>0$ and that the theorem is true for $n-1$

Then $\sum_{k \in \Z}^{n} {n \choose k} = \sum_{k \in \Z}^{n} [{n-1 \choose k-1} +{n-1 \choose k} $

=$ \sum_{k \in \Z}^{n} [{n-1 \choose k-1} + \sum_{k \in \Z}^{n} {n-1 \choose k} $

= $2^{n-1}+2^{n-1}$

= $2^n$

The Principal of Mathematical Induction the Theorem is proven

//

}

\item Combinatorial Argument Proof for Theorem 1.1

{PF

Notice that Pascal's Triangle has borders of all 1's

This is true because the coefficients of the first and last elements in Pascal's Triangle can be written as

${n \choose 0}$, where $n \in \Z$ for the first element.

The Last element can be written as ${n \choose n}$ where $n \in \Z$ 

${n \choose 0}$ and ${n \choose n}$ both = 1, hence every first and last element of Pascal's triangle is 1

Internally the numbers in Pascal's triangle are found by adding the two numbers above it

Because row two of Pascal's triangle is 1 1, the middle element of row three is 2

This can be written as ${n \choose k} = {n-1 \choose k-1} + {n-1 \choose k}$

Therefore to add an entire row of Pascal's triangle together we can use the formula $ \sum_{k=0 }^{n}{n \choose k} $

Notice $ \sum_{k=0 }^{n}{n \choose k} = 2^n$

Therefore the sum of the entries of row $n$ of Pascal's triangle is $2^n$


//
}

\item 1.1.29

{
PF

equation 1.14 = $(1+x)^n = {n \choose 0}+{n \choose 1}x^1+{n \choose 2}x^2.....{n \choose n}x^n$

$[(1+x)^n]' = n(1+x)^{n-1}$

$ [{n \choose 0}+{n \choose 1}x^1+{n \choose 2}x^2.....{n \choose n}x^n]' =  
[{n \choose 1}+{n \choose 2}2x+{n \choose 3}3x^2.....{n \choose n}nx^{n-1}$

Which can be written as $ n(1+x)^{n-1} =  \sum_{k=1 }^{n}k{n \choose k} x^{k-1}$


//
}





\end{enumerate}

\end{document}