\documentclass[12pt]{article}
\usepackage{amsmath}
\usepackage{amssymb}
\usepackage{graphicx}
\graphicspath{ {GraphTheory/ } }
 


\pagestyle{headings} \thispagestyle{empty}
%\pagestyle{empty}
  \textwidth      6.4in
      \oddsidemargin 0.0in
      \topmargin     -0.4in
      \topskip          0pt
      \headheight      12pt
      \footskip        18pt
%      \footheight      12pt
      \textheight     650pt

\parindent=0cm
\baselineskip=2cm

%\include these lines if you want to use the LaTeX "theorem" environments
\newtheorem{theorem}{Theorem}[section]
\newtheorem{definition}[theorem]{Definition}
\newtheorem{lemma}[theorem]{Lemma}
\newtheorem{corollary}[theorem]{Corollary}
\newtheorem{guess}{Conjecture}
\newtheorem{example}[theorem]{Example}

%include lines like this if you want to define your own commands
%to save typing
\newcommand{\PROOF}{\noindent {\bf Proof}: }
\newcommand{\REF}[1]{[\ref{#1}]}
\newcommand{\Ref}[1]{(\ref{#1})}
\newcommand{\dt}{\mbox{\rm   dt}}
\newcommand{\qed}{\Large{\bf{$\diamond$}}}
\newcommand{\phat}{\hat{p}}

\DeclareSymbolFont{AMSb}{U}{msb}{m}{n}
\DeclareMathSymbol{\N}{\mathbin}{AMSb}{"4E}
\DeclareMathSymbol{\Z}{\mathbin}{AMSb}{"5A}
\DeclareMathSymbol{\R}{\mathbin}{AMSb}{"52}
\DeclareMathSymbol{\Q}{\mathbin}{AMSb}{"51}
\DeclareMathSymbol{\I}{\mathbin}{AMSb}{"49}
\DeclareMathSymbol{\C}{\mathbin}{AMSb}{"43}

%\setstretch{1.5}

\renewcommand{\baselinestretch}{1.5}

\begin{document}


\textbf{Name: Taylor Heilman}    \hspace{4in} 
\begin{center} \textbf{CS 275: Spring 2016} \end{center}

{
1. 

Let $G$ be a connected graph

Consider a $u-v$ walk in $G$ of length $n$

Then there exists $v \in V(G)$ and there exists $u \in V(G)$

Since $G$ is connected there also exists a $u-v$ path of length $m$

Case 1: The $u-v$ walk and the $u-v$ path contain the same alternating sequence of vertices and edges as one another.
		
Since the $u-v$ walk and the $u-v$ path both share identical alternating sequences of vertices and edges

Then $m = n$ 

Case 2: The $u-v$ walk and the $u-v$ path contain different sequences of vertices as one another.

Then there exists a vertex $q \in$ the $u-v$ walk that is not in the  $u-v$ path

Therefore the length of the  $u-v$ walk, n, is made up of $d(u,q)$ and $d(q,v)$

By the definition of the triangle inequality length $m, d(u,v) \leq d(u,q) + d(q,v)$ for all $u,v,q$

Therefore, if a graph $G$ has a $u-v$ walk of length $n$, then $G$ has a $u-v$ path of length $m \leq n$


}

2.

{
Consider a graph $G$ 

Notice $rad(G)$ is the minimum eccentricity among all vertices in $G$ and $diam(G)$ is the maximum eccentricity among all vertices in $G$
	
Hence $rad(G) \leq diam(G)$

Suppose the distance between two vertices $u,v = diam(G)$ and the eccentricity of vertex $t = rad(G)$

By definition of the triangle inequality the $diam(G) \leq d(u,t) + d(v,t)$ because $diam(G)$ is the length of the shortest path between $u,v$, any other path from $u-v$ must be as long or longer than $diam(g)$

$d(u,t) + d(v,t) \leq rad(G) + rad(G)$  because $rad(G)$ is the maximum length of a path from $t$ to any other vertex $\in G$

Therefore $rad(G) \leq diam(G) \leq 2rad(G)$

}

3.
{
Consider a tree $T$

Suppose a pair of vertices $u,v \in V(T)$ are joined by a diametral path 

By definition of Theorem 4.2, a tree has wither 1 central vertex or 2 adjacent central vertices

Case 1: Graph T has 1 central vertex $c$

Suppose by contradiction that every diametral path in $T$ does not include all central vertices.

Hence there exists a $u-v$ path that does not contain c

Since $T$ is a tree we know the paths $c-u$ and $c-v$ exist

Therefore a $u-v$ path containing $c$ exists, which forms a contradiction with our hypothesis and also forms a cycle which cannot occur in a tree.

Case 2: Graph T has 2 adjacent, central vertices $c1, c2$
	
Suppose by contradiction that every diametral path in $T$ does not include all central vertices.

Hence there exists a $u-v$ path that does not contain $c1$ and $c2$

Since $T$ is a tree we know the paths $c1-u$ and $c2-v$ exist, since $c1$ is adjacent to $c2$ the path $u-v$ containing $c1, c2$ exists

Therefore a $u-v$ path containing $c1,c2$ exists, which forms a contradiction with our hypothesis and also forms a cycle which cannot occur in a tree.

Therefore in any tree $T$, every diametral path includes all central vertices of $T$



$t = ((25 * t + word[i] * word[0] + (len/2) $mod len) mod numslots


\end{document}