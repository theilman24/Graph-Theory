\documentclass[12pt]{article}
\usepackage{amsmath}
\usepackage{amssymb}
\usepackage{graphicx}
\graphicspath{ {GraphTheory/ } }
 


\pagestyle{headings} \thispagestyle{empty}
%\pagestyle{empty}
  \textwidth      6.4in
      \oddsidemargin 0.0in
      \topmargin     -0.4in
      \topskip          0pt
      \headheight      12pt
      \footskip        18pt
%      \footheight      12pt
      \textheight     650pt

\parindent=0cm
\baselineskip=2cm

%\include these lines if you want to use the LaTeX "theorem" environments
\newtheorem{theorem}{Theorem}[section]
\newtheorem{definition}[theorem]{Definition}
\newtheorem{lemma}[theorem]{Lemma}
\newtheorem{corollary}[theorem]{Corollary}
\newtheorem{guess}{Conjecture}
\newtheorem{example}[theorem]{Example}

%include lines like this if you want to define your own commands
%to save typing
\newcommand{\PROOF}{\noindent {\bf Proof}: }
\newcommand{\REF}[1]{[\ref{#1}]}
\newcommand{\Ref}[1]{(\ref{#1})}
\newcommand{\dt}{\mbox{\rm   dt}}
\newcommand{\qed}{\Large{\bf{$\diamond$}}}
\newcommand{\phat}{\hat{p}}

\DeclareSymbolFont{AMSb}{U}{msb}{m}{n}
\DeclareMathSymbol{\N}{\mathbin}{AMSb}{"4E}
\DeclareMathSymbol{\Z}{\mathbin}{AMSb}{"5A}
\DeclareMathSymbol{\R}{\mathbin}{AMSb}{"52}
\DeclareMathSymbol{\Q}{\mathbin}{AMSb}{"51}
\DeclareMathSymbol{\I}{\mathbin}{AMSb}{"49}
\DeclareMathSymbol{\C}{\mathbin}{AMSb}{"43}

%\setstretch{1.5}

\renewcommand{\baselinestretch}{1.5}

\begin{document}


\textbf{Name: Taylor Heilman}    \hspace{4in} 
\begin{center} \textbf{CS 275: Spring 2016} \end{center}




{
1. 

Let $T$ be the tree obtained from $G.BFS(v)$ (using the BFS algorithm on graph G with v as the root).

Suppose there is a vertex $w \in V(G)$ such that the shortest distance from $v$ to $w$ is of length n.

Let n = 1:

Then $w$ is adjacent to $v$, so $w \in C(v)$ and therefore a $(v,w)$ path of length 1 exists in $T$.

Assume $n = k$ and the minimum length of some $(v,w) path \in G$ is equal to the minimum length of the $(v,w) path \in T$

Let $n = (k+1)$

Then there exists some vertex $y \in G$ such that the shortest $(v,y)$ path in $G$ is of length (k+1).

According to out assumption, the following is true:

- Suppose some vertex, $e \in T$ exists with the distance of length 1 away from the root $v$. Following the BFS algorithm, $e$ gets an edge connecting itself to $v$. Thus a $v,e$ path has been created and has length 1.

- Suppose some vertex, $t \in T$ exists with the distance of length 2 away from $v$. Following the BFS algorithm, $t$ gets an edge connecting itself to $e$. Thus a $v,t$ path has been created and has length 2.

- We continue this procedure until we get to vertices with distances of length $k$ away from the root. Hence at least one vertex exists such that there is path of minimum length $k$  that connects $v$ to some vertex $z \in T$.

Now, since vertex $w$ has a $vw$ path of length k+1 in $G$, there exists a path of length $k$ to some vertex, $r$ adjacent to $w$ in $G$.  By creating an edge connecting $r,w$ a $v,w$ path is created and has length $k+1$ in $T$.

Therefore distance is preserved when using BFS.

}


{
2.

(WTS: $e_{T}(v) \leq e_{G}(v))$

Suppose there exists some graph $T$ which is a spanning tree of graph $G$.

Hence, $V(T) = V(G)$ but $|E(T)| \leq |E(G)|$.

So $T$, the spanning tree of $G$, has at most the same amount of edges as $G$ ($E(T) = E(G))$, or less edges than $G$.

Notice the $e_{G}(v)$ is simply MAX $d_{G}(v,w)$ where $w \in V(G)$.

Similarly, $d_{G}(v,w)$ is simply the shortest path between $v$ and $w$.

A $(v,w)$ path exists in T, and finding the path between $(v,w) \in T$ creates 2 possible cases.

Case 1: The path between $(v,w) \in T$ is the same as the path between $(v,w) \in G$.

In this case $d_{G}(v,w) = d_{T}(v,w)$ since they are the same path.

Case 2: The path between $(v,w) \in T$ is different than the path between $(v,w) \in G$.

In this case there exists some vertex $x$ which creates $vx$ and $xw$ paths $\in T$ which do not exist in the $(v,w)$ path $ \in G$.

By the Triangle inequality, the distance of the $vw$ path $\in T$ containing the vertex $x$  is $\geq$ to the distance of the $(v,w)$ path $ \in G$ which does not contain $x$.

Therefore $e_{T}(v) \leq e_{G}(v)$.
}




3.
{

1. Let  visited = [v] (v is the root)

2.Let  C = N(v) (adjacent vertices to v)

3. Let labels = [v] ('labels' is a list where the index of each vertex is its label number), p(v) =  v, b* = v (keeps track of current vertex we are branching from), i = 1.

4. For each $w \in N(b*)$ append w to labels.

5. Delete b* from all adjacency lists , remove b* from C and append b* to visited. Define a new b* to be the vertex in C such that $l[x]$ is minimum.  If C contains a vertex that is also in labels (this checks if a vertex adjacent to the current b* has already been labeled by a previous b*, which means a cycle has been found), return True.  Else return to step 4. If every vertex of $G$ is contained in visited return false.
}








\end{document}




